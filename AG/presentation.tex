\documentclass{beamer}

\usetheme{Pittsburgh}
\setbeamertemplate{navigation symbols}{}

\title{Wahrscheinlich eher unwahrscheinlich: Probabilistische Programmierung in Curry}
\author{Sandra Dylus\\Arbeitsgruppe Programmiersprachen und \"Ubersetzerkonstruktion}
\date{01. Juli 2015}


\begin{document}


\begin{frame}
\maketitle
\end{frame}


\begin{frame}{Probabilistische Programmierung}{Motivation}

\begin{enumeration}
\item ``Kann der Hamburger SV noch die Klasse halten?''
\item Wahrscheinlichkeit, mit der dieses Ereignis eintritt
\item ganz nat\"urliche Fragestellung im Kontext von Probablistischer Programmierung
\end{enumeration}
\end{frame}

\begin{frame}{Probabilistische Programmierung}{Grundlegendes}
\begin{enumeration}
\item zuf\"allige Werte als Primitive der Sprache
\item Definition von Probabilistischen Modellen
\item Spezifikation von bereits bekannt Aussagen (Einschr\"ankungen)
\item Berechnung von Anfragen an das Probabilistische Modell bzgl. dieser Einschr\"ankungen
\end{enumeration}
\begin{enumeration}
\item eingebettete DSL in vorhandener Programmiersprache (OCaml, Scala, Clojure, Haskell) 
\item neue Probabilistische Programmiersprachen (Church, Venture, Anglican, BLOG, ProbLog, PRISM, Stan, Tabular, Figaro, Infer.Net
\end{enumeration}
\end{frame}

\begin{frame}

\end{frame}
\end{document}

