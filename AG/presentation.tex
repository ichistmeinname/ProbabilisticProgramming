\documentclass{beamer}

\usetheme{Pittsburgh}
\setbeamertemplate{navigation symbols}{}

\title{Wahrscheinlich eher unwahrscheinlich: Probabilistische Programmierung in Curry}
\author{Sandra Dylus\\Arbeitsgruppe Programmiersprachen und \"Ubersetzerkonstruktion}
\date{01. Juli 2015}


\begin{document}


\begin{frame}
\maketitle
\end{frame}


\note{


Ich moechte in kurzer Form, meinen Vortrag in Bad Honnef als Einstieg
nutzen. %
Mein Vortrag in Bad Honnef beschaeftigte ich mich mit der Fragestellung,
ob zu dem damaligen Zeitpunkt der laufenden Saison, ein bestimmter
Tabellenplatz fuer den Hamburger SV noch potentiell moeglich war. %
Die grundlegende Idee dieser Fragestellung war keineswegs neu und
wurde in aehnlicher Form schon von Ingo Wegener im Rahmen von ein paar
Seminarbeiten und darauffolgenden Diplomarbeiten analysiert. %

Rahmenbedinungen fuer diese Fragestellung sind das Regelwerk des
Turniermodus der 1. Fussball Bundesliga. %
Zur Veranschaulichung habe ich die Tabelle aus meinem Buro
mitgenommen; kurz gesagt: es gibt 18 Vereine, die jeweils zweimal
gegeneinander spielen. Dabei unterscheidet man zwischen Hin- und
Rueckrunde, wobei jeweils die Heimmannschaft wechselt. %
Jede Mannschaft spielt also gegen jeder andere Mannschaft jeweils im
eigenen als auch im gegnerischen Stadion. %

Wir schauen uns kurz an, wie man diese Fragestellung samt
Rahmenbedinung in Curry modellieren kann. %
}
\begin{frame}{Motivation}{Wie alles begann}
\begin{minipage}{0.3\textwidth}
\begin{figure}
\includegraphics<1>[width=\textwidth]{images/curry-puzzle}
\caption{\footnotesize AG-Vortrag WS14/15: R\"atsell\"osen in Curry}
\end{figure}
\end{minipage}%
%
\hfill%
%
\begin{minipage}{0.6\textwidth}
\begin{figure}
\includegraphics<1>[width=\textwidth]{images/ligagott-zoom}
\caption{\footnotesize Bad Honnef Vortrag: Kann der Hamburger SV die Klasse halten?}
\end{figure}
\end{minipage}
\end{frame}


% %%%%%%%%%%%%%%%%%%%%%%%%%%%%%%%%%%%%%%%%%%%%%%%
%
%  Einschub: Bad Honnef
%
% %%%%%%%%%%%%%%%%%%%%%%%%%%%%%%%%%%%%%%%%%%%%%%%

%
%  Grundlegende Modellierung
%
\note{


Die genaue Implementierung jeder einzelnen Funktion, die hier
auftaucht, ist nicht relevant; vielmehr moechte ich schemenhaft zeigen,
wie ein derartiges Problem modelliert werden kann.

}
\begin{frame}[fragile]{Motivation}{Bad Honnef Vortrag}
Gibt es ausgehend von der aktuellen Tabelle eine Konstellation der
noch verbleibenden Spieltage, so dass mindestens zwei Mannschaften
(in der daraus resultierenden Tabelle) weniger
Punkte haben als der Hamburger SV?

\begin{semiverbatim}
question :: Table -> [Fixture] -> Bool
question table matchDays =
  thereExist 2 newTable
    `suchThat` all (`lessPointsThan` hsvPoints)
 where
  newTable = foldr updateTable table matchDays
  hsvPoints = points HamburgerSV newTable
\end{semiverbatim}

\end{frame}

%
%  Schluessel der Modellierung:
%    Simulation eines Spiels
%
\begin{frame}[fragile]{Motivation}{Bad Honnef Vortrag}

\begin{semiverbatim}
updateTable :: [Fixture] -> Table -> Table
updateTable fixtures table =
  foldr recalculateTable table results
 where
  results = map playFixture fixtures
\end{semiverbatim}

\begin{semiverbatim}
data Match = Match Team Team Result
data Result = HomeVictory | Draw | AwayVictory
\end{semiverbatim}

\begin{semiverbatim}
playFixture :: Fixture -> Match
playFixture (Fixture team1 team2) =
  Match team1 team2 _
\end{semiverbatim}


%
%  Daten fuer Modell sowie Anfrage
%
\note{
Ich fuettere mein Modell also mit Daten; in diesem Fall handelt es
sich um die verbleibenden drei Spieltage der vergangenen Saison. %
Als letzten Schritte stelle ich dann meine Frage. %
Zu diesem Zeitpunkt der Saison war ich in der Tat sehr gluecklich
darueber, dass ein Verbleib in der 1. Bundesliga noch moeglich war. %

}
\begin{frame}[fragile]{Motivation}{Bad Honnef Vortrag}

\begin{semiverbatim}
table =
  [(BayernMuenchen, 76), ...
  ,(WerderBremen, 42), ...
  ,(HamburgerSV,31),(SCPaderborn, 31),(SCFreiburg,30)
  ,(Hannover96,30),(VfBStuttgart, 27)]
\end{semiverbatim}

\begin{semiverbatim}
matchDay32 =
  [ Fixture HamburgerSV SCFreiburg
  , Fixture BayernMuenchen FCAugsburg
  , ...
  , Fixture Hannover96 WerderBremen ]
matchDay33 = [ ... ]
matchDay34 = [ ... ]
\end{semiverbatim}
  
\begin{semiverbatim}
> question table [matchDay32,matchDay33,matchDay34]
True
\end{semiverbatim}
  
\end{frame}
  
\end{frame}


% %%%%%%%%%%%%%%%%%%%%%%%%%%%
%
%
%
% %%%%%%%%%%%%%%%%%%%%%%%%%%%
\note{

Ich moechte nicht nur wissen, ob der gefragte Ausgang moeglich
ist -- sprich ``ja'' oder ``nein'' als Antwort, sondern mit welcher
Wahrscheinlichkeit das Ereignis eintritt. %

\begin{itemize}
\item Kann der Hamburger SV noch die Klasse halten?
\item Wahrscheinlichkeit, mit der Ereignis eintritt
\item ganz nat\"urliche Fragestellung im Kontext von Probabilistischer Programmierung
\end{itemize}

}
\begin{frame}{Motivation}{Probabilistische Programmierung}
\center
\Large
Wie hoch ist die Wahrscheinlichkeit, dass der Hamburger SV die Klasse
h\"alt?
\end{frame}

\begin{frame}{Probabilistische Programmierung}{Grundlegendes}
\begin{itemize}
\item zuf\"allige Werte als Primitive der Sprache
\item Definition von Probabilistischen Modellen
\item Spezifikation von bereits bekannten Ereignissen (Einschr\"ankungen)
\item Berechnung von Anfragen an das Probabilistische Modell bzgl. dieser Einschr\"ankungen
\end{itemize}
\begin{itemize}
\item eingebettete DSL in vorhandener Programmiersprache (OCaml, Scala, Clojure, Haskell) 
\item neue Probabilistische Programmiersprachen (Church, Venture, Anglican, BLOG, ProbLog, PRISM, Stan, Tabular, Figaro, Infer.Net)
\end{itemize}
\end{frame}



\begin{frame}{Probabilistische Programmierung}{Bayes'sches Netz}

\begin{itemize}
\item Beispiel (Graph)
\item Modellierung in Church/ProbLog
\item Modellierung in Curry
\end{itemize}
\end{frame}

\begin{frame}{Probabilistische Programmierung}{Ausblick und Fazit}

\begin{minipage}{.48\textwidth}
\begin{itemize}
\item Suchstrategien (Tiefensuche, Breitensuche, Iterative Deepening)
\item Lazy Evaluation mit Sharing
\item call-time-choice
\item Nichtdeterminismus mittels Darstellung als Suchbaum
\item freie Variablen
\end{itemize}
\end{minipage}
%
\hfill
%
\begin{minipage}{.48\textwidth}
\begin{itemize}
\item Inferenzalgorithmen
\item explizites Speichern von Berechnung via $mem$
\item \emph{Many World}-Semantik
\item BDD zur Darstellung des Suchbaumes
\item Lazy Auswertung f\"ur Suchbaum
\item Delimitted Control als Optimierung
\end{itemize}
\end{minipage}


\end{frame}

\end{document}